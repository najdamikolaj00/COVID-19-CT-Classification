\begin{thebibliography}{99.}

\bibitem{ITcovidReview}
Mondal S, Mitra P.: The Role of Emerging Technologies to Fight Against COVID-19 Pandemic: An Exploratory Review. Trans Indian Natl Acad Eng., 99--110 (2022)

\bibitem{MLDLcovidReview}
Showmick Guha Paul, Arpa Saha, Al Amin Biswas, Md. Sabab Zulfiker, Mohammad Shamsul Arefin, Md. Mahfujur Rahman, Ahmed Wasif Reza: Combating Covid-19 using machine learning and deep learning: Applications, challenges, and future perspectives. Array, Volume 17 (2023)

\bibitem{covidForecasting}
Rehman MU, Shafique A, Khalid S, Driss M, Rubaiee S.: Future Forecasting of COVID-19: A Supervised Learning Approach. Sensors (Basel), (2021)

\bibitem{sentimentTweets}
Alkhaldi, Nora A., Yousef Asiri, Aisha M. Mashraqi, Hanan T. Halawani, Sayed Abdel-Khalek, and Romany F. Mansour: Leveraging Tweets for Artificial Intelligence Driven Sentiment Analysis on the COVID-19 Pandemic. Healthcare 10, (2022)

\bibitem{MLDLimageReview}
Priya Aggarwal, Narendra Kumar Mishra, Binish Fatimah, Pushpendra Singh, Anubha Gupta, Shiv Dutt Joshi: COVID-19 image classification using deep learning: Advances, challenges and opportunities. Computers in Biology and Medicine, Volume 144 (2022)

\bibitem{zhao2020COVID-CT-Dataset}
Zhao, Jinyu and Zhang, Yichen and He, Xuehai and Xie, Pengtao: COVID-CT-Dataset: a CT scan dataset about COVID-19. arXiv preprint arXiv:2003.13865, (2020)

\bibitem{he2020sample}
He, Xuehai and Yang, Xingyi and Zhang, Shanghang, and Zhao, Jinyu and Zhang, Yichen and Xing, Eric, and Xie, Pengtao: Sample-Efficient Deep Learning for COVID-19 Diagnosis Based on CT Scans. medrxiv, (2020)

\bibitem{CNNexplanation}
Sakshi Indolia, Anil Kumar Goswami, S.P. Mishra, Pooja Asopa: Conceptual Understanding of Convolutional Neural Network- A Deep Learning Approach. Procedia Computer Science 132, 679--688 (2018)

\bibitem{CNNmedicalimageclassification}
Yadav, S.S., Jadhav, S.M.: Deep convolutional neural network based medical image classification for disease diagnosis. J Big Data 6, Volume 113 (2019)

\bibitem{TransferLearningCOVID19}
Kevser Sahinbas, Ferhat Ozgur Catak: Transfer learning-based convolutional neural network for COVID-19 detection with X-ray images. Data Science for COVID-19, 451-466 (2021)

\bibitem{SelfSupervisedLearnig}
Kriti Ohri, Mukesh Kumar: Review on self-supervised image recognition using deep neural networks. Knowledge-Based Systems, Volume 224, (2021)

\bibitem{PCA1}
Nandi, Dibyadeep Ashour, Amira S. Samanta, Sourav Chakraborty, Sayan Salem, Mohammed Abdel-Megeed Mohammed Dey, Nilanjan: Principal Component Analysis in Medical Image Processing: A Study. International Journal of Image Mining, 45-64 (2015)

\bibitem{PCA2}
Mateen, Muhammad Wen, Junhao Nasrullah, Dr Song, Sun Huang, Zhouping: Fundus image classification using VGG-19 architecture with PCA and SVD. Symmetry, (2018)

\bibitem{SVM1}
Khachane, Monali: Organ-Based Medical Image Classification Using Support Vector Machine. International Journal of Synthetic Emotions, 18-30 (2017)

\bibitem{SVM2}
Jair Cervantes, Farid Garcia-Lamont, Lisbeth Rodríguez-Mazahua, Asdrubal Lopez: A comprehensive survey on support vector machine classification: Applications, challenges and trends,
Neurocomputing, Volume 408, 189-215 (2020)

\bibitem{WeaklyFramework}
X. Wang et al.: A Weakly-Supervised Framework for COVID-19 Classification and Lesion Localization From Chest CT. IEEE Transactions on Medical Imaging, vol. 39, no. 8, pp. 2615-2625 (2020)

\bibitem{kfoldcrossvalidationN1000}
Baeldung on CS, \url{https://www.baeldung.com/cs/train-test-datasets-ratio}. Last accessed 19
March 2023

\bibitem{ImageNet}
J. Deng, W. Dong, R. Socher, L. -J. Li, Kai Li and Li Fei-Fei: ImageNet: A large-scale hierarchical image database. IEEE Conference on Computer Vision and Pattern Recognition, (2009)

\bibitem{LearningCurves}
Machine Learning Mystery, \url{https://machinelearningmastery.com/}. Last accessed 28
May 2023

\end{thebibliography}